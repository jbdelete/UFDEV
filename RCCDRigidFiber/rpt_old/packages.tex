% *** LANGUAGE PACKAGES ***
\usepackage[english]{babel} 
\usepackage[utf8]{inputenc}
\usepackage[T1]{fontenc}
\usepackage{lmodern}		% Great font
\renewcommand*\familydefault{\sfdefault}
\usepackage[useregional]{datetime2}		% To use several date formats
\usepackage{lipsum}		% For dummy text

% *** GEOMETRY PACKAGES ***
\usepackage{geometry}
\geometry{left=25mm,
	right=25mm,
	top=35mm,
	bottom=30mm,
	headheight = 35 mm
} 
\usepackage{lastpage}

% *** COLOR PACKAGES ***
\usepackage[table]{xcolor}		% "table" for table rowcolors
% Color definitiions
\definecolor{blue}{RGB}{0,89,140}
\definecolor{gray}{RGB}{242,242,242}
\definecolor{grayblack}{RGB}{50,50,50}
\definecolor{blue2}{RGB}{10,62,157}
\definecolor{red2}{RGB}{173,17,0}
\definecolor{gray2}{RGB}{230,230,230}

% *** HEADING AND FOOTER ***
\usepackage{fancyhdr} % For heading and footers
\renewcommand{\headrulewidth}{0.5pt}
\let\oldheadrule\headrule% Copy \headrule into \oldheadrule
\renewcommand{\headrule}{\color{blue}\oldheadrule}% Add colour to \headrule
\renewcommand{\footrulewidth}{0.5pt} 
\let\oldfootrule\footrule%
\renewcommand{\footrule}{\color{blue}\oldfootrule}% Add colour to \headrule
\pagestyle{fancy}                    % Default page style
\cfoot{}                             % Empty foot center  
\lhead{\includegraphics[width=0.15\textwidth]{Gatorlogo}}     
\chead{\textcolor{grayblack}{\author}}
\rhead{\textcolor{grayblack}{\DTMsetstyle{ddmmyyyy} \date}}
\lfoot{\textcolor{grayblack}{\small \title}}        
\rfoot{\textcolor{grayblack}{\small Page. \thepage\ - \pageref*{LastPage}}} 			     % Total of pages

% *** GRAPHICS RELATED PACKAGES ***
\usepackage{graphicx}       % Loading images
\usepackage{float}          % Figures inside minipages
\usepackage{wrapfig}		% Text wrapped around figure
\usepackage{tikz}			% Used to load cover figure
\usepackage[hypcap,font={color=grayblack}]{caption} % used to style the captions
\usepackage{subcaption} 	% For subfigures
\usepackage{overpic}		% To add text over figures
\graphicspath{{../images}}  % Figures relative directory

% *** TITLE PACKAGES ***
\usepackage{titlesec}
\titleformat{\section}{\color{blue}\normalfont\Large\bfseries}{\thesection}{1em}{}
\titleformat{\subsection}{\color{blue}\normalfont\large\bfseries}{\thesubsection}{1em}{}
\usepackage{setspace} % Para ajustar la separación entre líneas del documento
\usepackage{xfrac}
% *** TABLE PACKAGES ***
\usepackage{booktabs}
\usepackage{colortbl}
\usepackage{footnote} % To have footnotes inside tables
\usepackage{array}
\usepackage[printonlyused,withpage]{acronym}
\usepackage{xurl} % Lo cargo antes de hyperref, porque ese ya lo carga también.
\urlstyle{sf} % Estilo de los url pasa a Sans Serif.

\usepackage[colorlinks,
citecolor=cyan,
urlcolor=blue,
linkcolor=blue,
citebordercolor={0 0 1},
urlbordercolor={0 1 1},
linktocpage,
hyperfootnotes=true
]{hyperref}


%%% EQUATIONS %%%
\usepackage{mathtools}
%\usepackage[numbers]{natbib}
\usepackage[square,numbers]{natbib}
%\setReferenceFile{../bib/biblio}{../bst/IEEEtranSNLink}%
\bibliographystyle{../bst/IEEEtranSNLink}
%\usepackage[
%backend=biber,	% Backend para las referencias (no modificar)
%style=IEEEtranSN, 		% Estilo APA de bibliografía
%sortcites,		% Para tener ordenadas las citas
%natbib=true,	% Utiliza natbib
%url=true, 		% Para que aparezca o no la url
%doi=true,		% Para que aparezca o no el DOI
%isbn=false 		% Para que aparezca o no el ISBN
%]{biblatex}
%\addbibresource{../bib/biblio.bib}
\definecolor{Abstractcolor}{rgb}{0,0.6,0}
\definecolor{Commentcolor}{rgb}{0.74,0.11,0.16}
\definecolor{Objectcolor}{rgb}{0.6,0.0,0.4}
\newcommand{\ctxt}[1]{\textcolor{Commentcolor}{\\ \small QUOTE: #1 :ENDQUOTE \\}}
\newcommand{\q}[1]{\textcolor{red}{#1}}
\newcommand{\ncite}[1]{\textcolor{red}{CITE:#1}}
\newcommand{\todo}[1]{\textcolor{blue}{TODO:#1}}
\newcommand{\intendThe }[1]{\textcolor{blue}{GOAL:#1}}
\newcommand{\dens}{\mathfrak{m}}

%% JMB MY STUFF %%%%%%%%%%%%%%%%%%%%%%%%%%%%%%%%%%%%%%%%%%%%%%%%%%%%%%%%%%%%%
\definecolor{Abstractcolor}{rgb}{0,0.6,0}
\definecolor{Commentcolor}{rgb}{0.74,0.11,0.16}
\definecolor{Objectcolor}{rgb}{0.6,0.0,0.4}
\definecolor{mygreen}{rgb}{0,0.6,0}
\definecolor{mygray}{rgb}{0.5,0.5,0.5}
\definecolor{mymauve}{rgb}{0.58,0,0.82}
\definecolor{mybk}{rgb}{0.95,0.95,0.95}

\newcommand{\algo}[1]{Fig. #1}
\newcommand{\algn}[2]{Fig. #1, line #2}
\newcommand{\Algo}[1]{Fig. #1}
\newcommand{\Algn}[2]{Fig #1, line #2}
\newcommand{\figo}[1]{fig. \ref{fig:#1}}
\newcommand{\Figo}[1]{Fig. \ref{fig:#1}}
\newcommand{\figob}[2]{fig. \ref{fig:#1}-(#2)}
\newcommand{\Figob}[2]{Fig. \ref{fig:#1}-(#2)}
\newcommand{\app}{\textbf{\textit{rccdApp}}}
\newcommand{\ver}{\textit{\textbf{verApp}}}
\newcommand{\gen}{\textit{\textbf{genApp}}}
\newcommand{\mmrr}{\textit{\textbf{mmrr}}}
\newcommand{\marr}{\textit{\textbf{marr}}}
\newcommand{\arr}{\textit{\textbf{arr}}}
\newcommand{\mat}{\textit{\textbf{matApp}}}
\usepackage{listings}
\usepackage{verbatim}

\usepackage{verbatim}		
\usepackage{listings}			


\lstdefinestyle{gpucode}{
language=C,
keywordstyle=\color{blue},
commentstyle=\color{Commentcolor},
stringstyle=\color{darkgray},
numbers=left,
stepnumber=1,
numbersep=5pt,
numberstyle=\tiny,
breaklines=true,
breakautoindent=true,
breakatwhitespace=false,
frame=single,	          
title=\lstname,     
postbreak=\space,
tabsize=4,
basicstyle=\tiny, %\ttfamily\scriptsize,
showspaces=false,
showstringspaces=false,
extendedchars=true,
backgroundcolor=\color{white},
 morekeywords=[2]{layout},
 morekeywords=[3]{binding},
 morekeywords=[4]{uniform},
 morekeywords=[5]{location},
 morekeywords=[6]{in},
 morekeywords=[7]{out},
 morekeywords=[8]{vec3},
 morekeywords=[9]{vec2},
 morekeywords=[10]{vec4},
 morekeywords=[11]{mat4},
 morekeywords=[12]{gl_Position},
 morekeywords=[13]{version},
 morekeywords=[14]{float},
 morekeywords=[15]{Particle},
 morekeywords=[16]{inout},
 morekeywords=[17]{uint},
 morekeywords=[18]{WIDTH},
 keywordstyle = [2]\color{blue},
 keywordstyle = [3]\color{blue},
 keywordstyle = [4]\color{blue},
 keywordstyle = [5]\color{blue},
 keywordstyle = [6]\color{blue},
 keywordstyle = [7]\color{blue},
 keywordstyle = [8]\color{violet},
 keywordstyle = [9]\color{violet},
 keywordstyle = [10]\color{violet},
 keywordstyle = [11]\color{violet},
 keywordstyle = [12]\color{violet},
 keywordstyle = [13]\color{green},
 keywordstyle = [14]\color{violet},
 keywordstyle = [15]\color{ForestGreen},
 keywordstyle = [16]\color{blue},
 keywordstyle = [17]\color{blue},
 keywordstyle = [18]\color{mymauve}
 }
\include{../code/code_C.tex}

\lstnewenvironment{queryl}[1][] 
 {\lstset{frame=shadowbox,escapechar=`,linewidth=8cm, #1}}
 {}