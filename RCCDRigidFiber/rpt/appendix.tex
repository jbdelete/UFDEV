% !TeX root = main.tex
\newpage
\clearpage
\section{Appendix}


\subsection{Particle-Particle Collision/Resolution CodingDetails}\label{ppcrc}

Collision resolution is processed in the compute pipeline as shown in the code of \figo{ParticleComp}. The \textit{potentially colliding set} is delivered to the compute kernel by the graphics pipeline. Starting at line 2, the code iterates over the eight corners of the \textit{axis aligned bounding box} (AABB) encasing the particle sphere, seeking other particle corners that occupy the same cell. The code traverses the length of the array which contains all of the particles in the cell at line 12. When a particle is found, it is registered in a duplicates array, since two or more particle corners can span the same cells. If it is not a duplicate the \texttt{ProcessParticleContact(..)} function is called in \figo{ProcessParticleContact}. 
   

\begin{figure}[h]
\centering
\lstset{style=gpucode,linewidth=6.5in,xleftmargin=0.25in}

\begin{lstlisting}
///..........
	for(uint ii = 0; ii < 8;ii++)
	{
		// Set location to local variable.
		uint loc = P[Findex].zlink[ii].ploc;

		// If the lcation is not zero..
		if (loc != 0)
		{
			// Use the linked particle location to index into the particle-cell hash table 
			// And compare this particle with all of the paricles at this location.
			for(uint jj = 0; jj< MAX_ARY;jj++)
			{
				Tindex = clink[loc].idx[jj];

				// If the linked particle is zero teminate
				if(Tindex == 0)
					break;
				
				
					
				
				for(uint ii = 0; ii <= dupcnt && dupcnt < 256; ii++)
				{
					if(duplst[ii] == Tindex)
					{
						dupflg = true;
						break;
					}
					if(duplst[ii] == 0)
					{
						dupflg = false;
						duplst[ii] == Tindex;
						dupcnt = ii;
						break;
					}
				}
										
				if(dupflg == false)						
				{
					ProcessCDBoundary(Findex,Tindex,OutVel);
					ProcessParticleContact(ii,Findex, Tindex, OutVel);
				}
			}
		}
	}
///..........
\end{lstlisting}


\caption[Benchset test configuration file]{A snippet of the compute kernel which processes the PCS }
\label{fig:ParticleComp}
\end{figure}


The code in \figo{ProcessParticleContact} performs \textit{contact determination} by using a squared distance determination between centers (line 28) and a squared distance of the sum of their radii (line 33). If the squared the distance between centers is less than the squared distance of the sum of their radii then they are in contact. The squared distances are used because the square root function is expensive on the GPU. If the particles are in contact then the resulting momentum reaction is calculated.


\begin{figure}[h]
\centering
\lstset{style=gpucode,linewidth=6.5in,xleftmargin=0.25in}

\begin{lstlisting}
// Takes the index of two particles and detemines the distance between them
// If the distance is less than the sum of radii squared the are in comllsion.
// If collsiong increment the collsion counter.
uint ProcessParticleContact(uint crnr, uint Findex, uint Tindex, in out vec3 OutVel)
{
	if(Findex == Tindex || Tindex <= bbound)
		return 0;
	
	vec3 U1x,U1y,U2x,U2y,V1x,V1y,V2x,V2y;

	float xT = P[Findex].PosLoc.x;
    float yT = P[Findex].PosLoc.y;
    float zT = P[Findex].PosLoc.z;
	
	float xP = P[Tindex].PosLoc.x;
    float yP = P[Tindex].PosLoc.y;
    float zP = P[Tindex].PosLoc.z;
	
	float Fm 	= P[Findex].MolarMatter;
	float Ft 	= P[Tindex].MolarMatter;
	vec3 InPosF = P[Findex].PosLoc.xyz;
	vec3 InPosT = P[Tindex].PosLoc.xyz;
	vec3 InVelF	= P[Findex].prvvel.xyz;
	vec3 InVelT	= P[Tindex].prvvel.xyz;
	vec3 newVel;
	
    // Get squared distance between centers
    float dsq = ((xP-xT)*(xP-xT)+
                    (yP-yT)*(yP-yT)+
                    (zP-zT)*(zP-zT));
   
	// Get sqaured diameter
	float rsq = ((P[Findex].PosLoc.w+P[Tindex].PosLoc.w)*(P[Findex].PosLoc.w+P[Tindex].PosLoc.w));
	
	// If square of distance is less than square of radii there is a collision.
	if (dsq <= rsq )
    {
		// This particle has collision
		P[Findex].ColFlg = 1;
		
		// Make sur this is not a duplicate collision.
		if(inColl(Findex,Tindex))
		{
			return 0;
		}
		// Calulate the resolution
		CalcMomentum(Findex,Fm,Ft,InPosF,InPosT,InVelF,InVelT,newVel);
		P[Findex].VelRad.xyz = newVel;
		// Count collisions
		atomicAdd(collOut.CollisionCount,1);
		return -1;		
	}
	else
	{
		// Not in collsion anymore
		P[Findex].ColFlg = 0;
		// Clear dup flags
		ClearCflg(Findex,Tindex);
	}


	return 0;
}
\end{lstlisting}


\caption[Benchset test configuration file]{The code for \texttt{ProcessParticleContact(...)} does a distance based contact determination. }
\label{fig:ProcessParticleContact}
\end{figure}



\begin{figure}[h]
\centering
\lstset{style=gpucode,linewidth=6.in,xleftmargin=0.25in}

\begin{lstlisting}
uint CalcMomentum(	uint Findex, 			// Source particle index
					float Fm, 			    // Source mass		
					float Ft,				// Target mass
					vec3 InPosF,			// Source position
					vec3 InPosT,			// Target position
					vec3 InVelF,			// Source velocity
					vec3 InVelT,			// Target velocity
					in out vec3 newVel)		// Returned new velocity
{

	float m1, m2, x1, x2;
	vec3 v1temp, v1, v2, v1x, v2x, v1y, v2y; 
	vec3 x = InPosT-InPosF;

	//Process source particle 
	x = normalize(x);
	v1 = InVelF;
	x1 = dot(x,v1);
	v1x = x * x1;
	v1y = v1 - v1x;
	m1 = Fm;
	
	//Process target particle 
	x = x*-1;
	v2 = InVelT;
	x2 = dot(x,v2);
	v2x = x * x2;
	v2y = v2 - v2x;
	m2 = Ft;

	//Return velocity for source particle
	newVel = vec3( v1x*(m1-m2)/(m1+m2) + v2x*(2*m2)/(m1+m2) + v1y );
    return 0;
}
\end{lstlisting}


\caption[Calculate collision momentum]{The code for \texttt{CalcMomentum} which performs a 3-d reflection around the line of impact between two particles. }
\label{fig:CalcMomentum}
\end{figure}


\subsection{Particle-Boundary Collision/Resolution CodingDetails}\label{pbcrc}

Next, in the same manner as particle-particle collision the squared distance between the particle and the wall is determined and if the the particle lay within it the boundary collision it is processed (line 33) by calling the \texttt{CalcBoundaryVel(..)} function, \figo{CalcBoundaryVel}. 


\begin{figure}[h]
\centering
\lstset{style=gpucode,linewidth=6.5in,xleftmargin=0.25in}

\begin{lstlisting}
void ProcessCDBoundary(uint Findex, uint Bindex, in out vec3 OutVel)
{

	// Process this particle only if it is a particle, not a boundary particle.
	// particle 
	if(Bindex > bbound)
		return;
		
		
	float tol = 0.5;
	vec3 InPosF = P[Findex].PosLoc.xyz;
	vec3 InVelF	= P[Findex].VelRad.xyz;
	vec3 InPosB;
	vec3 InVelB;
	
	// The positon of the target boundary particle is stored in the velocity.
	// If any of these components are non-zero then we are in a cell with a boundary.
	if(P[Bindex].VelRad.x != 0.0 || P[Bindex].VelRad.y != 0.0 || P[Bindex].VelRad.z != 0.0 ) 
	{
		float xT = P[Findex].PosLoc.x;
		float yT = P[Findex].PosLoc.y;
		float zT = P[Findex].PosLoc.z;
		// Get the boundary radius at this point.
		float radius = GetCDRadius(P[Findex].PosLoc.z);
		// Square it.
		float dsq = radius*radius;
	    // Get the squared position of the source particle.
		float yr = abs(yT-CENTER)+2*P[Findex].PosLoc.w;
		float xr = abs(xT-CENTER)+2*P[Findex].PosLoc.w;
		float psq = ((yr*yr)+(xr*xr));	
	
		// If we are within the squared radius of the bpoundary process it.
		if(psq >= dsq && P[Findex].bcs[0].clflg == 0)
		{
			
			newVel = CalcBoundaryVel(Findex,InPosF,InVelF,CENTER);
			// Set the new velocity.
			P[Findex].VelRad.xyz = newVel;
			// Set the "in collsion" flag.
			P[Findex].bcs[0].clflg = 1;
		}
		else
		{	// Give it 6 frames to get out of range of the bouddary so it does not
			// repeat the collsion. We could 'impulse' out of the boudary but then we 
			// may knock another particle out of the boudary. THis is hardcoded for this 
			// demonstration only.
			if(P[Findex].bcs[0].clflg++ == 6)
				P[Findex].bcs[0].clflg = 0;
		}

	}
	
}

\end{lstlisting}


\caption[Benchset test configuration file]{The code for \texttt{ProcessCDBoundary(...)} which performs a distance calculation for a boundary particle. }
\label{fig:ProcessCDNBoundary}
\end{figure}


\Figo{CalcBoundaryVel} shows the code for the calculation of equations (\ref{eqn:bound001},\ref{eqn:bound002}). There are some other considerations, such as contact with a flat. 

If the collision is with a flat, then the \texttt{GetCDRadius(...)} function returns the negative of its length (line 19). If this is the case, only the sign of the xy velocity components need be changed (lines 25-28). If the particle is on the slope of the nozzle then the three independent points are determined as described. Care must be taken when getting independent points on the slope, to insure that the reach does not cross over into another section of the nozzle that is off the slope in question (lines 40-50).


\begin{figure}[h]
\centering
\lstset{style=gpucode,linewidth=6.5in,xleftmargin=0.25in}

\begin{lstlisting}
vec3 CalcBoundaryVel(uint index, vec3 Pos, vec3 Vel, uint Center) 
{
	uint startf = 4465;
	uint endf = 4470;
	uint particl = 4704;
	
	vec3 pointA;
	vec3 pointB;
	vec3 pointC;
	float radA;
	float radB;
	float lowZ;
	vec3 rvel;
	
	Get the angle of the particle in the XY plane.
	float angxy = atan2piPt(vec2(Pos.x-Center,Pos.y-Center));
	
	// Get the radius at this point.
	radA = GetCDRadius(Pos.z);
	
	// If the radius is negative we are on a flat 
	// so just revers xyvelocity
	if(radA < 0.0)
	{
		rvel.x = -Vel.x;
		rvel.y = -Vel.y;
		rvel.z = Vel.z;
		return rvel;
	}

	// Get point A at the same angle of the particle but
	// on the boundary
	pointA.x = radA*cos(angxy)+Center;
	pointA.y = radA*sin(angxy)+Center;
	pointA.z = Pos.z;

	// For the next point we need a vector up or doen the slope
	// parallel to Point A. THis code makes sure we always get
	// that point from another part of the nozzle that is sloping.
	if(Pos.z >= secx2_beg && Pos.z  < secx2_end - 10) 
		lowZ = pointA.z+4.0;
		
	else if(Pos.z >= secx2_beg+10.0 && Pos.z < secx2_end)
		lowZ = pointA.z-4.0;
		
	if(Pos.z >= secx4_beg && Pos.z < secx4_end - 10 ) 
		lowZ = pointA.z+4.0;
		
	else if(Pos.z >= secx4_beg+10.0 && Pos.z < secx4_end )
		lowZ = pointA.z-4.0;	
	
	// Get the new radius for point B and calulate the vector.
	radB = GetCDRadius(lowZ);
	pointB.x = radB*cos(angxy)+Center;
	pointB.y = radB*sin(angxy)+Center;
	pointB.z = lowZ;

	// These two previous point are not independent since they
	// are parallel we need a point off to the side. Rotate
	// the point B vactor a little and esablish the last 
	// indepented point on the plane    
	pointC.x = radB*cos(angxy+PI/64)+Center;
	pointC.y = radB*sin(angxy+PI/64)+Center;
	pointC.z = lowZ;
		
	// Get the plane normal
	vec3 param1 = pointA-pointB;
	vec3 param2 = pointA-pointC;
	vec3 normvec = cross(pointA-pointB, pointA-pointC);
	float D = -dot(normvec,	pointA);
    vec3 nnormvec = normalize(normvec);
	
	// Caclulate velocity reflection.
	rvel = Vel - 2.0*(dot(Vel,nnormvec)*nnormvec);

	return rvel;

}
\end{lstlisting}


\caption[Benchset test configuration file]{The code for \texttt{CalcBoundaryVel(...) } which performs the calculates required to reflect the particle around the boundary norm.}
\label{fig:CalcBoundaryVel}
\end{figure}


The final component of boundary detection is the function that returns the radius at any point along the z direction of the nozzle \figo{GetCDRadius}. This is a piecewise function that returns a positive value if the particle is on the slope of the boundary and a negative value if it is on a flat. If this function represented an equation then it could not only return the radius of the boundary at any point, but the tangent plane by taking the derivative.   


\begin{figure}[h]
\centering
\lstset{style=gpucode,linewidth=6.5in,xleftmargin=0.25in}

\begin{lstlisting}
const float sec24slp = 0.1f;

const float secx1_beg = 1.0f;
const float secx1_end = 5.0f;
const float secz1     = 10.0f;

const float secx2_beg = 5.0f;
const float secx2_end = 25.0f;
const float secz2     = 10.0f;

const float secx3_beg = 25.0f;
const float secx3_end = 30.0f;
const float secz3     = 8.1f;


const float secx4_beg = 30.0f;
const float secx4_end = 50.0f;
const float secz4     = 10.0f;

const float secx5_beg = 50.0f;
const float secz5     = 10.0f;

float GetCDRadius(float Z)
{
	// SEC1 flat return negative of radius
	if(Z >= secx1_beg && Z <= secx1_end)
       return -secz1;
	   
	// SEC2 slope down
	// 5.0 included to less than 25
	else if(Z > secx1_end && Z <= secx2_end)
	   return secz2+(Z-secx2_beg)*(-sec24slp);

	// SEC3 flat return negative of radius
	else if(Z > secx2_end && Z <= secx3_end)
       return -secz3;
	   
	// SEC4 slope up
	else if(Z > secx3_end && Z <= secx4_end)
		return secz3+(Z-secx3_end)*(sec24slp);
		
	// SEC5 flat
	else if(Z > secx4_end)
        return -secz5;  
		
	else
		return 0;

	return 0;
}
\end{lstlisting}


\caption[Benchset test configuration file]{The code for \texttt{GetCDRadius(...)} which returns the radius of the nozzle at any point along its z-length.}
\label{fig:GetCDRadius}
\end{figure}


\subsection{Update Position}\label{up}

\Figo{ChangePosCDNoz} shows the code for position change. A number of checks are performed here, starting with determining if the particle exceeded the boundary(lines 21-25). If this happens the particle is disabled. The next checks are to determine if this run has motion enabled by configuration file, if the particle has been disabled, or if the simulation has been stopped or started by pressing the 'S' key (line 29).

The angle of the yz components of velocity are taken (lines 34-36) and are stored in the particle variable (line 37). This value is inserted in the first component of the HSV color map. It should be noted that the spherical angles of the velocity can also be determined and inserted into the first two components of the HSV color map. Additionally, even though the angles are taken in a slice of the yz plane, the 3-D dimension of the particle velocity can be plotted. For this demonstration we are only using the yz plane. 

With all of the calculus being performed \textit{not a number} (nan) can be returned. This is a major error and causes the simulation to shut down. Finally, if the particle has exited the nozzle it is disabled. 

\begin{figure}[h]
\centering
\lstset{style=gpucode,linewidth=6.5in,xleftmargin=0.25in}

\begin{lstlisting}
uint ChangePosCDNoz(uint index)	
{		
	
	// There was an error with this particle do not process it anymore
	if(P[0].PosLoc.w == 1.0)
		return 1;

	// Get radius squared to check to see if the
	// particle has violated a boundary
	float radius = GetCDRadius(P[index].PosLoc.z);
	
	float dsq = radius*radius;
	float yT = P[index].PosLoc.y;
	float xT = P[index].PosLoc.x;
	float yr = (yT-CENTER);
	float xr = (xT-CENTER);
	float psq = ((yr*yr)+(xr*xr));	
	
	// If the positon of the particle is beyond a boundary
	// and it has not already been reported  disable it
	if(psq > dsq && uint(P[index].prvvel.w) == 0)
	{
		P[index].prvvel.w = 1.0;
		return 1;
	}
	
	// Calulate change in position if we are configured for motion,
	// and the particle is active, and the stop flag is not engaged
	if(doMotion == 1 && uint(P[index].prvvel.w) == 0 && uint(ShaderFlags.StopFlg) == 0)
		P[index].PosLoc.xyz += P[index].VelRad.xyz*dt;
	
	
	// Calculate the yz angle
	vec2 angnorm = normalize(P[index].VelRad.zy);			
	float angletmp = atan2piPt(angnorm); 
	// HSV Normalize and assign to particle angle variable
	P[index].FrcAng.w = atan2piPt(angnorm)/(2*PI);

	// Can't have nan's, abort simulation.
	if(isnan(P[index].FrcAng.w) || isnan(P[index].VelRad.x) || isnan(P[index].VelRad.y) || isnan(P[index].VelRad.z))
	{
		collIn.ErrorReturn = 7;
		collIn.particleNumber = index;
	}
	// Stop once the particle is beyond the nozzle
	if(P[index].PosLoc.z > 64.4)
		P[index].prvvel.w = 1.0;
	return 0;
	
}	 
\end{lstlisting}


\caption[Benchset test configuration file]{The code for \texttt{ChangePosCDNoz(...)} which calculates displacement after the PCS has been evaluated for collisions in the compute kernel. }
\label{fig:ChangePosCDNoz}
\end{figure}


In this demonstration a total of 156,924 particles exhibited no boundary violations or nan's.
