
\begin{figure}[h]
\centering
\lstset{style=gpucode,linewidth=6.in,xleftmargin=0.25in}

\begin{lstlisting}
uint CalcMomentum(	uint Findex, 			// Source particle index
					float Fm, 			    // Source mass		
					float Ft,				// Target mass
					vec3 InPosF,			// Source position
					vec3 InPosT,			// Target position
					vec3 InVelF,			// Source velocity
					vec3 InVelT,			// Target velocity
					in out vec3 newVel)		// Returned new velocity
{

	float m1, m2, x1, x2;
	vec3 v1temp, v1, v2, v1x, v2x, v1y, v2y; 
	vec3 x = InPosT-InPosF;

	//Process source particle 
	x = normalize(x);
	v1 = InVelF;
	x1 = dot(x,v1);
	v1x = x * x1;
	v1y = v1 - v1x;
	m1 = Fm;
	
	//Process target particle 
	x = x*-1;
	v2 = InVelT;
	x2 = dot(x,v2);
	v2x = x * x2;
	v2y = v2 - v2x;
	m2 = Ft;

	//Return velocity for source particle
	newVel = vec3( v1x*(m1-m2)/(m1+m2) + v2x*(2*m2)/(m1+m2) + v1y );
    return 0;
}
\end{lstlisting}


\caption[Calculate collision momentum]{The code for \texttt{CalcMomentum} which performs a 3-d reflection around the line of impact between two particles. }
\label{fig:CalcMomentum}
\end{figure}
