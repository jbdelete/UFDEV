
\begin{figure}[h]
\centering
\lstset{style=gpucode,linewidth=6.5in,xleftmargin=0.25in}

\begin{lstlisting}
const float sec24slp = 0.1f;

const float secx1_beg = 1.0f;
const float secx1_end = 5.0f;
const float secz1     = 10.0f;

const float secx2_beg = 5.0f;
const float secx2_end = 25.0f;
const float secz2     = 10.0f;

const float secx3_beg = 25.0f;
const float secx3_end = 30.0f;
const float secz3     = 8.1f;


const float secx4_beg = 30.0f;
const float secx4_end = 50.0f;
const float secz4     = 10.0f;

const float secx5_beg = 50.0f;
const float secz5     = 10.0f;

float GetCDRadius(float Z)
{
	// SEC1 flat return negative of radius
	if(Z >= secx1_beg && Z <= secx1_end)
       return -secz1;
	   
	// SEC2 slope down
	// 5.0 included to less than 25
	else if(Z > secx1_end && Z <= secx2_end)
	   return secz2+(Z-secx2_beg)*(-sec24slp);

	// SEC3 flat return negative of radius
	else if(Z > secx2_end && Z <= secx3_end)
       return -secz3;
	   
	// SEC4 slope up
	else if(Z > secx3_end && Z <= secx4_end)
		return secz3+(Z-secx3_end)*(sec24slp);
		
	// SEC5 flat
	else if(Z > secx4_end)
        return -secz5;  
		
	else
		return 0;

	return 0;
}
\end{lstlisting}


\caption[Benchset test configuration file]{The code for \texttt{GetCDRadius(...)} which returns the radius of the nozzle at any point along its z-length.}
\label{fig:GetCDRadius}
\end{figure}
